\documentclass{article}


\begin{document}
%\begin{table}[ht]
%\caption {GROUP MEMBERS}
%\centering
%\begin{tabular}{c c c c }
%\hline
%NAMES ANDREG NO
%\hline
% MUBIRU GRACE 15/U/7729/PS\\
% MARVINN KAFUKO 15/U/5419/PS\\
% DAVIS KABISWA 15/U/318 \\
%EDWINS DILLA 14/U/6131/PS\\
%\hline
%\end{tabular}
%\label{table:nonlin}
%\end{table}

\title{A CONTEST DESIGN RESEARCH CONCEPT PAPER}
\maketitle

\section {INTRODUCTION }{A ”contest” refers to a widespread form of competition that has attracted a
huge iterature in Economics. In a contest, players expend costly effort that
translate to some form of output, or score. Then they (or some of them) receive
prizes associated with their ranking on output/score }

\subsection {Background to the problem} {Contest design has got its roots in three models developed from the seventies to the eighties. These models include Tullock(1980) model of rent seeking, lazear and rosen (1981)  rank-order tournament model and all pay auction( Hirshleifer and Riley 197 8; ).The most
interesting thing about these models is that they can be well studied as games
where different agents act as players with their respective efforts affecting their
respective probabilities of winning. They have also been studied extensively by economic theorists in what as become known has the field of contest theory (konrand 2009) In this research therefore, we are going to
introduce contest design as games with contestants doing whatever it takes to
win. }

\subsection {Problem Statement} {
The project would study a class of contests considered in a paper ”Ranking games that have competitiveness-based strategies”. 
} 


\subsection {Aim and Objectives}{
\subsubsection *{\begin{small}general objective\end{small}}{One aspect of the project would be to study the convergence properties of the Fictitious Play procedure, applied to these games.}
\subsubsection *{\begin{small}Specific Objectives\end{small}}{We also envisage addressing, experimentally, the challenge of handicapping.“Handicapping involves ranking the players on the values
of monotonic functions of their outputs, rather than just raw outputs, in order to elicit greater effort”;
to choose the best functions}
}
\subsection {Research Scope}  {This research targets the economic, political and social environments which can be described as contests in which the competing agents have the opportunity to expend the scarce resources like money, time, etc. in order to affect the probabilities of winning prizes.The project is experimental, with some scope for analytical work (specifically, for the problem of optimal handicapping in the 2-player case).


\section {Research Significance}   {This study is useful because it reviews studies investigating the basic structure of contests, including the number of players and prizes, spillovers and externalities, heterogeneity, risk and incomplete information. Second, we discuss dynamic contests and multi-battle contests }
\section {References}{\textit {konradm K A 2009,"Strategy and dynamics in contests.New York", Oxford University Press}\\
{\textit {Tullock, G. (1980). Efficient rent seeking. In James M. Buchanan, Robert D. Tollison,  Gordon Tullock}\\
{Toward a theory of the rent-seeking society (pp. 97 – 112). College Station, TX: Texas A and M University Press.}
}\\
{\textit {Lazear, E. P., and Rosen, S. (1981). Rank-order tournaments as optimum labor contracts.
Journal  of
Political Economy, 89
, 841–864.
}\\
{\textit {Hirshleifer, J., and Riley, J. G. (1978). Elements of the theory of auctions and contests. UCLA, Working
Papers.
}
}
}

\end{document}