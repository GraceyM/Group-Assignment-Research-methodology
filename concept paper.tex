\documentclass{article}
\begin{document}
\begin{table}[ht]
\caption{GROUP MEMBERS}
\centering
\begin{tabular}{c c c c}
\hline
NAMES & REGN NO.\\
\hline
MUBIRU GRACE &15/U/7729/PS \\
MARVINN KAFUKO &15/U/5419/PS \\
DAVIS KABISWA & 15/U/318 \\
EDWINS DILLA & 14/U/6131/PS \\ 
\hline
\end{tabular}
\label {table:nonlin}
\end{table}
\textbf{A CONTEST DESIGN RESEARCH CONCEPT PAPER }
\section{\textbf{INTRODUCTION}}
A "contest" refers to a widespread form of competition that has attracted a huge iterature in Economics. In a contest, players expend costly effort that translate to some  form of output, or score. Then they (or some of them) receive prizes associated with their ranking on output/score. 
Many economic, political and social environments can be described as contests in which the competing agents have the opportunity to expend the scarce resources like money, time, etc. in order to affect the probabilities of winning prizes.
As is obvious from this list, these environments have attracted considearable in attention in applications in a wide range of fields outside economics.
Contest design has got its roots in three models developed  from the seventies to the eighties. These models include rent seeking, rank-order tournament and all pay auction.The most interesting thing about these models is that they can be well studied as games where different agents act as players with their respective efforts affecting their respective probabilities of winning.
In this research therefore, we are going to introduce contest design as games with contestants doing whatever it takes to win.

l
\section{\textbf{PROBLEM STATEMENT}}
The project would study a class of contests considered in a paper "Ranking games that have competitiveness-based strategies". One aspect of the project would be to study the convergence properties of the Fictitious Play procedure, applied to these games. We also envisage addressing, experimentally, the challenge of handicapping. Handicapping involves ranking the players on the values of monotonic functions of their outputs, rather than just raw outputs, in order to elicit greater effort; the challenge is to choose the best functions. The project is experimental, with some scope for analytical work (specifically, for the problem of optimal handicapping in the 2-player case).






\end{document}